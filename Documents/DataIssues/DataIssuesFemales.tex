\documentclass{article}

\usepackage[
a4paper, % Paper size
top=1in, % Top margin
bottom=1in, % Bottom margin
left=1in, % Left margin
right=1in, % Right margin
%showframe % Uncomment to show frames around the margins for debugging purposes
]{geometry}
\title{First questions on female data}


\begin{document}
	\maketitle
	\section*{Currently impossible to fit+forecast}
	
	
\begin{itemize}
	\item too few data
	\begin{itemize}
	\item 30) French Polynesia: 3 available years are too few (unless I modified the whole forecasting model)
	\item 54) Moldova: 4 available years are too few (unless I modified the whole forecasting model)
	\item 64) Peru: 3 available years are too few (unless I modified the whole forecasting model)
	\item 67) Quatar: 4 available years and old ones (2010 2011 2012 2014) are too few (unless I modified the whole forecasting model)
	\end{itemize}
	\item 2019 not available (likely solvable if 2019 is interpolated as previous not-available years)
	\begin{itemize}
	\item 19) Cuba: missing 2019, what to do?
	\item 88) Uruguay: missing 2019, what to do?
	\end{itemize}
\end{itemize}
	
	\section*{Issues with infant mortality}
	
	\begin{itemize}
	\item Bermuda: no infant deaths for all pre-pandemic years but 2014 (~impossible to have a infant-soecialized coeff)
	\item Faroe Islands: no infant deaths in last years
	\item Liechtenstein: absence of infant deaths in the last period causes some issue (solvable?) 
	\item Montserrat: no infant deaths for all pre-pandemic years but 2015 (~impossible to have a infant-specialized coeff)
	\item Turks and Caicos Islands: No infant deaths for all pre-pandemic years but 2015 (~impossible to have a infant-specialized coeff)
	\item Tuvalu: no infant deaths in the first available years (~impossible to have a infant-specialized coeff)
	
	\end{itemize}
	
	\section*{Evident strange age-patterns (it could be solvable, but it denotes issues in actual data)}
	
	\begin{itemize}
		\item Albania: strange age-pattern at highest ages and increasing infant mortality
		\item Aruba: strange increasing of infant mortality. Actually death at age 0 in 2012 and 2014
		\item Azerbaijan: odd rapid increase of oldest-age mortality
		\item Bosnia and Herzegovina: decreasing observed mortality at higher ages (I can force monotonicity over age, but weird data)
		\item Bulgaria: wiggling age-pattern in first years at old ages (force monotonicity over age?)
		\item Croatia: wiggling age-pattern in first years at old ages (force monotonicity over age?)
		\item Lesotho: special treatment is needed (would enforcement of monotonicity be enough?)
		\item Lithuania: wiggling age-pattern at old ages (force monotonicity over age?)
		\item Maldives: too strong leveling-off at odest ages and odd data in 2019 (missing 2018)
		\item Mauritius: very weird 2011 data
		\item Montenegro: decreasing observed mortality at higher ages (I can force monotonicity over age, but weird data)
		\item New Zeland: a bit too strong leveling-off at oldest ages in last observed years
		\item North Macedonia: wiggling age-pattern at old ages (force monotonicity over age?)
		\item Romania: odd rapid increase of oldest-age mortality
		\item Serbia: a bit too strong leveling-off at oldest ages in all observed years
		\item South Korea: wiggling/decreasing age-pattern at old ages (force monotonicity over age?)
		\item Taiwan: optimal smoothing parameter produces wiggling age-pattern, likely need to impose extra-smoothness
		\item Turks and Caicos Islands: 3 available years, the fit is possible because they cover 5 years and include 2019. 
		\item Ukraine: optimal smoothing parameter produces wiggling age-pattern, likely need to impose extra-smoothness
	\end{itemize}
	
	
	\section*{Issues only in computing e0}
	\begin{itemize}
	\item due to large open-age group
		\begin{itemize}
\item Belize: extremely large open-age interval, 65+ => issues in e0 computation
\item Iran: extremely large open-age interval, 65+ => issues in e0 computation
\item Thailand: extremely large open-age interval, 65+ => issues in e0 computation
\end{itemize}
	\item general issue (due to GC lifetable code?) also because the fit on log-rates seems OK
		\begin{itemize}
	\item Colombia: good fit on rates, but likely issues in computing e0
\item Costa Rica: good fit on rates, but likely issues in computing e0
\item Ecuador: good fit on rates, but likely issues in computing e0
\item Honk-Kong: good fit on rates, but likely issues in computing e0
\item Japan: good fit on rates, but likely issues in computing e0
\item Malasya: good fit on rates, but likely issues in computing e0
\item Oman: good fit on rates, but likely issues in computing e0
\item Panama: good fit on rates, but likely issues in computing e0
\item State of Palestine: good fit on rates, but likely issues in computing e0
\item Suriname: good fit on rates, but likely issues in computing e0
\item Uzbekistan: good fit on rates, but likely issues in computing e0
\end{itemize}
	
	\end{itemize}
	
\section*{General questions from GC:}
	
	\begin{itemize}
	\item Iceland: data available only from 2013?
\item Italy: only by age-group? and from 2011?
	\end{itemize}

	

\end{document}


%\documentclass{beamer}
%\usepackage[table]{xcolor}
%
%\definecolor{maroon}{cmyk}{0,0.87,0.68,0.32}
%
%\begin{document}
%	
%	\noindent\begin{tabular}{|l|c|}
%		\rowcolor[HTML]{EFEFEF} %\rowcolor{maroon}
%		one & two \\
%%		\rowcolor{maroon!50}
%		three & four \\
%%		\rowcolor{maroon!10}
%		five & six
%	\end{tabular}
%	
%\end{document}

%\documentclass{article}
%
%\usepackage[backend=biber,style=alphabetic,]{biblatex}
%
%\title{A bibLaTeX example}
%
%\addbibresource{sample.bib} %Imports bibliography file
%
%\begin{document}
%	\section{First section}
%	
%	Items that are cited: \textit{The \LaTeX\ Companion} book \cite{latexcompanion} together with Einstein's journal paper \cite{einstein} and Dirac's book \cite{dirac}---which are physics-related items. Next, citing two of Knuth's books: \textit{Fundamental Algorithms} \cite{knuth-fa} and \textit{The Art of Computer Programming} \cite{knuth-acp}.
%	
%	\medskip
%	
%	\printbibliography
%\end{document}


%\documentclass[12pt,a4paper]{article}
%
%\usepackage{filecontents}

%\begin{filecontents*}{mybib.bib}
%	@book{aristotle:physics,
%		options    = {useauthor=false,usetranslator=true},
%		author       = {Aristotle},
%		title        = {Physics},
%		date         = 1929,
%		translator   = {Wicksteed, P. H. and Cornford, F. M.},
%		publisher    = {G. P. Putnam},
%		location     = {New York},
%		keywords     = {primary},
%		langid       = {english},
%		langidopts   = {variant=american},
%		shorttitle   = {Physics},
%		annotation   = {A \texttt{book} entry with a \texttt{translator} field},
%	}
%\end{filecontents*}
%
%
%\usepackage[backend=bibtex,giveninits=true,style=authoryear-ibid]{biblatex} 
%\addbibresource{mybib.bib}
%
%\begin{document}
%	
%	\cite{aristotle:physics}
%	
%	\printbibliography
%\end{document}
%\documentclass{article}
%\usepackage{amsmath,bm}
%\usepackage{multirow}
%\usepackage[table,xcdraw]{xcolor}
%\begin{document}
%%	% ...
%%	\[\arraycolsep=0pt\def\arraystretch{2}
%%	\begin{array}{ccccc}
%%	x(1) & = & \dfrac{x(0)}{1} & = & x(0)\\
%%	x(2) & = & \dfrac{x(1)}{2} & = & \dfrac{x(0)}{2}\\
%%	x(3) & = & \dfrac{x(2)}{3} & = & \dfrac{x(0)}{2.3}\\
%%	x(4) & = & \dfrac{x(3)}{4} & = & \dfrac{x(0)}{4!}\\
%%	&  &  &  & \vdots\\
%%	x(n) & = & \dfrac{x(n-1)}{n!} & = & \dfrac{x(0)}{n!}
%%	\end{array}
%%	\]
%%	% ...
%%	
%%	\bigskip
%%	
%%	
%%	\[\arraycolsep=1pt\def\arraystretch{1.2}
%%	\begin{array}{rll}
%%	x(1) &= \dfrac{x(0)}{1} &= x(0)\\
%%	x(2) &= \dfrac{x(1)}{2} &= \dfrac{x(0)}{2}\\
%%	x(3) &= \dfrac{x(2)}{3} &= \dfrac{x(0)}{2.3}\\
%%	x(4) &= \dfrac{x(3)}{4} &= \dfrac{x(0)}{4!}\\
%%	&                  & \vdots\\
%%	x(n) &= \dfrac{x(n-1)}{n!} &= \dfrac{x(0)}{n!}
%%	\end{array}
%%	\]
%%	
%%	
%%	
%%	
%%	\[\arraycolsep=1pt\def\arraystretch{1.2}
%%			\left[
%%	\begin{array}{c}
%%	\bm{\alpha}^{(i+1)}\\
%%	\bm{\nu}^{(i+1)}
%%	\end{array}\right] =
%%	\left[
%%	\begin{array}{c}
%%	\bm{\alpha}^{(i)}\\
%%	\bm{\nu}^{(i)}
%%	\end{array}\right] - 	
%%	 \left[ \begin{array}{ccc}
%%	\bm{L}^{(i)}  &:& \bm{H}^{(i)}\\
%%	\bm{H}^{(i)'}  &:& \bm{0}
%%	\end{array}	\right]^{-1}
%%		\left[ \begin{array}{c}
%%	\bm{r}^{(i)} - \bm{L}^{(i)}\bm{\alpha}^{(i)}\\
%%	\bm{\delta}(\bm{\alpha}^{(i+1)})
%%	\end{array}\right]
%%	\]
%%	
%
%\begin{table}[]
%	\begin{tabular}{c|l|l|l}
%		Model                           & à comparer avec & $\quad$ $\Delta$ AIC $\quad$& âge de séparation \\ \hline\hline
%		& \cellcolor[HTML]{EFEFEF}Weibull    & \cellcolor[HTML]{EFEFEF}          & \cellcolor[HTML]{EFEFEF}              \\
%		& Kannisto                           &                                   &                                       \\
%		& \cellcolor[HTML]{EFEFEF}Beard      & \cellcolor[HTML]{EFEFEF}          & \cellcolor[HTML]{EFEFEF}              \\
%		\multirow{-4}{*}{Gompertz}      & Log-quadratic                      &                                   &                                       \\ \hline
%		& \cellcolor[HTML]{EFEFEF}Gompertz   & \cellcolor[HTML]{EFEFEF}          & \cellcolor[HTML]{EFEFEF}              \\
%		& Kannisto                           &                                   &                                       \\
%		& \cellcolor[HTML]{EFEFEF}Beard      & \cellcolor[HTML]{EFEFEF}          & \cellcolor[HTML]{EFEFEF}              \\
%		\multirow{-4}{*}{Weibull}       & Log-quadratic                      &                                   &                                       \\ \hline
%		& \cellcolor[HTML]{EFEFEF}Gompertz   & \cellcolor[HTML]{EFEFEF}          & \cellcolor[HTML]{EFEFEF}              \\
%		& Weibull                            &                                   &                                       \\
%		& \cellcolor[HTML]{EFEFEF}Beard      & \cellcolor[HTML]{EFEFEF}          & \cellcolor[HTML]{EFEFEF}              \\
%		\multirow{-4}{*}{Kannisto}      & Log-quadratic                      &                                   &                                       \\ \hline
%		& \cellcolor[HTML]{EFEFEF}Gompertz   & \cellcolor[HTML]{EFEFEF}          & \cellcolor[HTML]{EFEFEF}              \\
%		& Weibull                            &                                   &                                       \\
%		& \cellcolor[HTML]{EFEFEF}Kannisto   & \cellcolor[HTML]{EFEFEF}          & \cellcolor[HTML]{EFEFEF}              \\
%		\multirow{-4}{*}{Beard}         & Log-quadratic                      &                                   &                                       \\ \hline
%		& \cellcolor[HTML]{EFEFEF}Gompertz   & \cellcolor[HTML]{EFEFEF}          & \cellcolor[HTML]{EFEFEF}              \\
%		& Weibull                            &                                   &                                       \\
%		& \cellcolor[HTML]{EFEFEF}Kannisto   & \cellcolor[HTML]{EFEFEF}          & \cellcolor[HTML]{EFEFEF}              \\
%		\multirow{-4}{*}{Log-quadratic} & Beard                              &                                   &                                      
%	\end{tabular}
%\end{table}
%	
%\end{document}
%
%%\documentclass{article}
%%
%%\usepackage{longtable}
%%\usepackage{lipsum} % just for dummy text- not needed for a longtable
%%
%%\begin{document}
%%	\lipsum[1]
%%	\lipsum[1]
%%	\lipsum[1]
%%	
%%	%\begin{table}[h] 
%%	%\centering
%%	\begin{longtable}{| p{.20\textwidth} | p{.80\textwidth} |} 
%%		\hline
%%		foo & bar \\ \hline 
%%		foo & bar \\ \hline
%%		foo & bar \\ \hline
%%		foo & bar \\ \hline
%%		foo & bar \\ \hline
%%		foo & bar \\ \hline
%%		foo & bar \\ \hline
%%		foo & bar \\ \hline
%%		foo & bar \\ \hline
%%		foo & bar \\ \hline
%%		foo & bar \\ \hline
%%		\caption{Your caption here} % needs to go inside longtable environment
%%		\label{tab:myfirstlongtable}
%%	\end{longtable}
%%
%%Table \ref{tab:myfirstlongtable} shows my first longtable.
%%\end{document}
%%\documentclass[12pt]{article}
%%
%%\usepackage[a4paper,left=2.3cm,right=2.3cm,bottom=2.3cm,top=2.8cm,footskip=32pt]{geometry}
%%\usepackage{rotating}
%%%\usepackage{euler}
%%\usepackage{makeidx}
%%\usepackage[ansinew]{inputenc}
%%\usepackage{graphicx}
%%\usepackage{epsfig}
%%\usepackage{fancybox}
%%\usepackage{rotating}
%%\usepackage{boxedminipage}
%%\usepackage{fancyhdr}
%%\usepackage{url}
%%\usepackage[sort]{natbib}
%%\usepackage{amsmath, amssymb, bm}
%%
%%\begin{document}
%%	
%%	\pagestyle{fancy} 
%%	\lhead{\textsl{Event History Analysis: Breakout Rooms}} 
%%	\rhead{\includegraphics[width=0.70cm]{INEDLogoSHORTissimo.jpg}$\,$
%%		\includegraphics[width=0.92cm]{MPIDRLogoNEW.jpg}$\,$
%%		\includegraphics[width=1.8cm]{EDSDLogoLarge.jpg}\textsl{EDSD 2020/21}} 
%%	\title{\normalsize{\textit{Event History Analysis - EDSD 2020/21}} \\$\,$ \\ 
%%		\Huge{Breakout Rooms for the lab-sessions}}
%%	\author{Giancarlo Camarda}
%%	\date{}
%%	
%%	\maketitle \thispagestyle{empty}
%%	%%%%%%%%%%%%%%%%%%%%%%%%%%%%%%%%%%%%%%%%%%%%%%%%%%%%%%%%%%%%%%%%%%
%%\maketitle
%%	
%%	
%%\subsubsection*{Room 1:}
%%\begin{itemize}\itemsep-0.5em
%%	\item Lucas Pitombeira
%%	\item Gianluca Superti        
%%	\item Liliana Patricia Calder\'on Bernal
%%	\item Marilyn-Anne Tremblay
%%	\item Anna Vera J{\o}rring Pallesen
%%\end{itemize}
%%
%%\subsubsection*{Room 2:}
%%\begin{itemize}\itemsep-0.5em
%%	\item Gonzalo Daniel Garcia
%%	\item Donata Stonkute
%%	\item Ursula Gazeley
%%	\item Ainhoa-Elena Leger
%%	\item Josephine Ackah
%%	\item Silvia Gaston-Guiu
%%\end{itemize}
%%
%%\subsubsection*{Room 3:}
%%\begin{itemize}\itemsep-0.5em
%%	\item Alon Pertzikovitz
%%	\item \"Ozer Bakar
%%	\item Rafael Navarro
%%	\item Alice Wolfle
%%	\item Su Yeon Jang
%%	\item \"Ozge Elif \"Ozer
%%\end{itemize}
%%
%%
%%\subsubsection*{Room 4:}
%%\begin{itemize}\itemsep-0.5em
%%	\item Carlos F\'elix Vega
%%	\item Silvio Cabral
%%	\item Milena Che{\l}chowska
%%	\item Paola V\'azquez-Castillo
%%	\item Arno Muller
%%\end{itemize}
%%
%%
%%
%%%$$
%%%e_{x} = \int_{x}^{\infty}e^{-\int_{x}^{a} \mu(y) \diff y} \diff a
%%%$$
%%%blabla \cite{CamardaJSSpackage2012}
%%%
%%%\bibliographystyle{chicago}
%%%\bibliography{bibliografia1}
%%
%%%	$$
%%%	\begin{array}{rrrr}
%%%	8 &\times& 1 = &8\\
%%%	8 &\times& 2 = &16\\
%%%	8 &\times& 3 = &24\\
%%%	8 &\times& 4 = &32\\
%%%	8 &\times& 5 = &40\\
%%%	8 &\times& 6 = &48\\
%%%	8 &\times& 7 = &56\\
%%%	8 &\times& 8 = &64\\
%%%	8 &\times& 9 = &72\\
%%%	8 &\times& 10 = &80
%%%	\end{array}
%%%	$$
%%%
%%%\bigskip
%%%\bigskip
%%%\bigskip
%%%\bigskip
%%%\bigskip
%%%\bigskip
%%%
%%%	$$
%%%\begin{array}{rrrr}
%%%9 &\times& 1 = &9\\
%%%9 &\times& 2 = &18\\
%%%9 &\times& 3 = &27\\
%%%9 &\times& 4 = &36\\
%%%9 &\times& 5 = &45\\
%%%9 &\times& 6 = &54\\
%%%9 &\times& 7 = &63\\
%%%9 &\times& 8 = &72\\
%%%9 &\times& 9 = &81\\
%%%9 &\times& 10 = &90
%%%\end{array}
%%%$$
%%%\end{Huge}
%%%	
%%
%%
%%\end{document}
%
%
%
%%
%%%% \documentclass[10pt,xcolor={dvipsnames}]{beamer}
%%%% \setbeamertemplate{navigation symbols}{}
%%%% \usepackage{caption}
%%%% \usepackage{subcaption}
%%%% \usepackage{url}
%%
%%%% \author{The author}
%%%% \title{Package conflicts}
%%%% \date{October 3rd, 2012}
%%
%%%% \begin{document}
%%
%%%% This is supposed to be \textcolor{OliveGreen}{olivegreen}
%%%% \end{document}
%%\documentclass{article}
%%\usepackage{tcolorbox}                                 %
%%\definecolor{lightblue}{rgb}{0.145,0.6666,1} % Defines the color used
%%% for content box headers 
%%\definecolor{Red}{rgb}{1,0.05,0.05}
%%%% \definecolor{Blue}{rgb}{0.1,0.3,1}
%%\definecolor{Blue}{rgb}{0.1,0.7,1}
%%\definecolor{Green}{rgb}{0.3,0.8,0.15}
%%\definecolor{Lightgray}{rgb}{0.86,0.86,0.86}
%%\definecolor{DarkGreen}{rgb}{0,0.4,0}
%%\definecolor{Orange}{rgb}{1,0.2,0}
%%\usepackage[sort]{natbib}
%%\author{The author}
%%\title{Package conflicts}
%%\date{}
%%
%%\begin{document}
%%
%%\begin{table}[]
%%\centering
%%\begin{tabular}{l|l|l|l}
%%\begin{tabular}[c]{@{}l@{}}Entropy:\\ \\ $H = \frac{-\int_{10}^{\omega} [{\color{Orange}S(x)}]^2 {\color{Orange}S(x)} dx}{\int_{10}^{\omega} {\color{Orange}S(x)} dx}$\end{tabular}                            & \raisebox{-3cm}{\includegraphics[scale=0.3]{fSWEfit}}   & \begin{tabular}[c]{@{}l@{}}Gini coefficient:\\ \\ $H = \frac{-\int_{10}^{\omega} [{\color{Orange}S(x)}]^2 {\color{Orange}S(x)} dx}{\int_{10}^{\omega} {\color{Orange}S(x)} dx}$\end{tabular} & FigG   \\ \hline
%%\begin{tabular}[c]{@{}l@{}}Stand. dev. above the mode:\\ \\ $SD(M+) = \sqrt{\frac{\int_{\hat{M}}^{\omega} (x-\hat{M})^2 {\color{blue} f(x)},dx}{\int_{\hat{M}}^{\omega} {\color{blue} f(x)} dx}}$\end{tabular} & FigSDM & \begin{tabular}[c]{@{}l@{}}Rate of aging:\\ \\ $H = \frac{-\int_{10}^{\omega} [{\color{Orange}S(x)}]^2 {\color{Orange}S(x)} dx}{\int_{10}^{\omega} {\color{Orange}S(x)} dx}$\end{tabular}    & FigRoA
%%\end{tabular}
%%\end{table}
%%
%%% \maketitle
%%
%%% blabla \cite{giancarlo} 
%% 
%%
%%$$
%%X_i \stackrel{iid}{\sim} N(\mu, \sigma^2), i={1, 2, ..., n} 
%%$$
%%
%%$$
%%f(x_i|\mu,\sigma^2) = \frac{1}{\sqrt{2\pi}\sigma}\textrm{exp}( {-\frac{(x_i-\mu)^2}{2\sigma^2}} ) $$
%%
%%\bigskip
%%$$
%%X_i \stackrel{iid}{\sim} N(\mu, \sigma^2), i={1, 2, ..., n}
%%$$
%% 
%%$$
%%f(x_i|\mu,\sigma^2) = \frac{1}{\sqrt{2\pi}\sigma}\textrm{exp}(
%%{-\frac{(x_i-\mu)^2}{2\sigma^2}} )
%%$$
%%
%%$$
%% f(x_1, x_2, ..., x_n|\mu,\sigma^2) =
%% \prod_{i=1}^{n}f(x_i|\mu,\sigma^2) \textrm{ by independence} 
%%$$
%%
%%$$
%% L(\mu,\sigma^2|x_1,x_2,...,x_n) = f(x_1, x_2, ..., x_n|\mu,\sigma^2) \]
%%$$
%%
%%% \bibliographystyle{chicago}
%%% \bibliography{bibliografia}
%%
%%\end{document}
%%
%%
%%
%%
%%
%%
%%
%%%% \documentclass{article}
%%%% %% \usepackage[utf8]{inputenc}
%%%% \usepackage[english]{babel}
%% 
%%%% \usepackage[usenames, dvipsnames]{color}
%% 
%%%% \begin{document}
%% 
%%%% This example shows different examples on how to use the \texttt{color} package 
%%%% to change the colour of elements in \LaTeX.
%% 
%%%% \begin{itemize}
%%%% \color{ForestGreen}
%%%% \item Firts item
%%%% \item Second item
%%%% \end{itemize}
%% 
%%%% \noindent
%%%% {\color{RubineRed} \rule{\linewidth}{0.5mm} }
%% 
%%%% The background colour of some text can also be \textcolor{red}{easily} set. For 
%%%% instance, you can change to orange the background of \colorbox{BurntOrange}{this 
%%%% text} and then continue typing.
%% 
%%%% \end{document}
%%
%%% \documentclass{article}
%%
%%% \usepackage[sort]{natbib}
%%
%%% \newcommand{\myindex}[1]{#1 \index{#1}}
%%
%%% \newcommand{\myindice}[1]{#1 \index{#1@\texttt{#1}}}
%%
%%% \usepackage{makeidx}
%%
%%% \makeindex
%%
%%% \begin{document}
%%
%%% % To solve various problems in physics, it can be advantageous
%%% % to express any arbitrary piecewise-smooth function as a \myindex{Fourier Series}
%%% % composed of multiples of sine and cosine functions.
%%
%%% % \myindex{\texttt{mean()}}
%%
%%% % \myindex{\texttt{zzz()}}
%%
%%% \myindex{mmm}
%%
%%% \myindex{vvv()@\texttt{vvv()}}
%%
%%% \texttt{Lin()} \index{Lin()@\texttt{Lin()}} 
%%
%%% \texttt{\myindex{lon()}}
%%
%%% \myindice{kkk()}
%%
%%% kkk()
%%
%%% % \index{hello}	hello, 1	Plain entry
%%
%%% % \index{hello!Peter}	  Peter, 3	Subentry under 'hello'
%%
%%% % \index{hello!Sam@\textsl{Sam}}	  Sam, 2	Subentry
%%% % formatted and sorted
%%
%%% % \index{Sam@\textsl{Sam}}	Sam, 2	Formatted entry
%%
%%% % \index{Jenny|textbf}	Jenny, 3	Formatted page number
%%
%%% % \index{Joe|textit}	Joe, 5	Same as above
%%
%%% % \index{ecole@\'ecole}	école, 4	Handling of accents
%%
%%% % \index{Peter|see {hello}}	Peter, see hello
%%% % Cross-references
%%
%%% % \index{Jen|seealso{Jenny}}
%%
%%% % blablabla \cite{AslanidouEtAlBayesMultivSurvMC1998} blablabla
%%% % blablabla \citep{BiatatCurrieJointModel2010} blablabla
%%% % \bibliographystyle{chicago}
%%% % \bibliography{bibliografia}
%%
%%
%%% \printindex
%%
%%  
%%   
%%
%%
%%
%%
%%% \end{document}
%%
%%
%%
%%%% Dear Viorela,
%%
%%%% thanks for the abstracts and for effort you made to this submission.
%%
%%%% In general I'm not sure you could talk about tempo effect here. This
%%%% concept has a peculiar meaning in demography and unless you introduce
%%%% it and link it to our ideas, it'd be hard to justify its presence. 
%%%% Here we are more into the concept of standardization as a way to
%%%% compare conceptually similar patterns which span over rather different
%%%% age-range. I am not sure that while standardizing we remove ``tempo effect''.
%%%% Of course I may have missed something, so please write me back if you don't
%%%% agree.
%%
%%%% Below some comments that loosely follow the paper structure.
%%
%%%% Hope it helps and all the bets,
%%%% Giancarlo 
%%
%%%% short abstract:
%%
%%%% ``by absolute and standardized by modal age at death scales''
%%
%%%% no novel about the nonparametric approach
%%
%%%% no need to specify in the short-abstract
%%%% ``specifically adapted to the context of cause of death analysis.''
%%
%%%% I would better write: ``by using a nonparametric approach such the P-splines.'' 
%%
%%%% I would add  ``As instance, preliminary results...''
%%
%%
%%%% long abstract:
%%
%%%% In general, I agree with NO (I'm writing offline, I do not know whether she
%%%% already replied) that section on methods is (1) unbalanced with
%%%% respect to the novelty on standardization which appear secondary (2)
%%%% there is not a clear reference to our previous work in its first
%%%% part. I wouldn't subdivide the section on method so, at least, the
%%%% standardization part will not seem too short.
%%
%%%% I agree with RB about the title (and all the other comments): the
%%%% current one is fine. 
%% 
%%%% I am not convinced about the adverb ``Successively,'' in the
%%%% introduction. It seems to me that genes and behavior have always an
%%%% important role in probability of surviving.
%%
%%%% Not sure about verb ``stipulated'' on Buffon's work. 
%%
%%%% on the absolute and relative to the associated mode time scale
%%
%%%% I'd specify ``We focus only on six leading causes...''
%%
%%
%%%% I would add k as superscript of the standardized age. 
%%
%%%% I'd specify more that with our approach we basically shift the chronological age
%%%% such that all distributions regardless their peculiar features are
%%%% place with their modal age at 1.     
%%
%%
%%%% what do you mean with ``K=k'' in the fist paragraph of the theoretical
%%%% model section?
%%
%%%% I would re-phrase the first part of the section on estimation: at
%%%% least in the abstract no need to define death rates and the
%%%% approximation of the force of mortality by Thatcher. My English is not
%%%% great, so fell free to modify, but I would write something like:
%%
%%%% A descriptive analysis of cause-specific mortality age-patterns could
%%%% be carried out by computing death rates. On one hand this approach is
%%%% attractive for its simplicity, on the other hand it does not allow to
%%%% precisely compute density functions and associated modal ages at death.      
%%
%%%% We cope with these issues by assuming smooth mortality patterns over
%%%% ages and thus retaining the continuity of mortality presented above.
%%
%%%% For a give calendar year and sex, let d^k_i represent observed death
%%%% counts for age i and cause k. In the following we will focus on
%%%% mortality over age 10 disregarding causes of death which are distinct
%%%% of infant mortality.
%%
%%%% We assume d^k_i as Poisson distributed data:
%%%% d^k_i ~ Poi(\mu^k(x) e)
%%%% where e denote the population’s amount of exposure to the risk of dying for
%%%% single year of age, and \mu^k(x) is assumed to smoothly
%%%% varying over x.
%%
%%%% Among various smoothing techniques, we opted for the $P$-splines which
%%%% has proved to be a highly effective approach for smoothing mortality
%%%% rates and consequently to obtain smooth forces of mortality (Currie et
%%%% al., 2004; Camarda, 2008). 
%%
%%%% [...]
%%
%%%% shorten the sentence afterward with something like:
%%
%%%% Once smooth cause-specific density functions are obtained, we estimate
%%%% the modal age at death for each cause by numerically computing the
%%%% correspondent density (Ouellette and Bourbeau, 2011). 
%%
%%%% The B-spline basis is not cause-specific, I guess, so delete its
%%%% superscript k.
%%
%%
